\chapter{Introduction}
A three-dimensional musical instrument might sound quite abstract for the bystander. One can think of it as a musical instrument taking place in the virtual reality or augmented reality domain. While an exact definition might be hard to settle because every instrument will be different in core features to others, a general definition might be an instrument which can have either :
\begin{itemize}
\item A visual representation in a three-dimensional space
\item Interactions in a three-dimensional space
\end{itemize}

The two points are generally shared, however it is harder to display the instrument in 3D than to interact with it.

The display can have two goals :  \cite{berthaut2010piivert}
\begin{itemize}
\item Giving visual cues to the spectators of the musician's actions.
\item Helping the musician to perform.
\end{itemize}

One of the main focuses of this exposé will be to assess the different display techniques suitable to a 3D instrument, and the other will explain how it can improve existing 3D instruments.

The last part of this report will be about the choices we had to make in order to set-up our own 3D musical instrument.
\chapter{Conclusion}
This project was the opportunity to discover a part of computer science that is cross-domain with optics and electronics in one part, and art and music in the other part.

We were not able to complete every required part of the implementation, mainly due to the complexity of the Drile code, however we still managed to have a working software for the aerial percussion.

There is quite a lot of room for improvement, however. 
First, on the artistic side, the software can be made to look more appealing to the eyes, for instance by using textures and particles.
Also, for the Pepper's Ghost technique, the Plexiglas screen could be replaced with a transparent tarpaulin : it seems it gives better results and it would also allow for a bigger projection size.
Also, with more time and budget, it could be possible to try to implement the render for other types of displays than two-view, for instance multi-view or omni-view. Lasers were proposed but due to the short amount of time, we couldn't try working with them.

Finally, it would be interesting to try with a two-computer setup to have spectator and performer view : this would be the best way to use the Drile.
\section{History of the 3D musical instruments}
De nombreux instruments de musiques immersifs se concentre sur la navigation dans un environnements 3D virtuelle.
Tout d'abord le projet Phase \cite{rodet2005study} explore la génération, la prise en main et le controle de son ou de musique à l'aide d'un capteur haptique et d'une représentation visuelle pouvant guidée l'utilisateur.
Un second projet, Plumage \cite{plumage2007}, est une interface pour le contrôle interactif de la composition audio spatialisées. Des plumes dispersées dans une scène 3D représente des grains sonores, générent du son lorsque des têtes de lectures les parcours. Les têtes de lectures sont contrôlées directement par l'utilisateur.
Néanmoins ces deux projets ne permettent pas de manipuler directement la structure de la synthèse sonore, mais seulement de la manipuler.
\\
Une autre gamme d'instrument 3D se concentre sur une unique synthèse sonore. Dans ce cas nous pouvons trouver par exemple le \it{Virtual Xylophone}, le \it{Virutal Membrane} ou encore la \it{Virtual Air Guitar} \cite{maki2005}.
\section{History of the 3D musical instruments}
\subsection{3D Navigation instruments}
Many immersive musical instruments tend to focus on navigation in a 3D virtual environment.
First of all, there is the Phase project  \cite{rodet2005study} which explores generation, handling and control of sound (or music), through a haptic sensor and a visual representation that guides the user.
Secondly, there is Plumage \cite{plumage2007} which is an interface for interactive control of \glslink{spatialization}{spatialized} audio composition. Feathers are scattered in a 3D scene which represents sound grains. They generate sound when they are crossed by reading heads. Those heads are controlled by the user.
Nevertheless these two projects do not allow manipulation of direct sound synthesis structure, only of existing sounds. 

\subsection{Single sound synthesis instruments}
Another range of 3D instruments focuses on a single kind of sound synthesis. For instance, the \textit{Virtual Xylophone}, the \textit{Virtual Membrane} or the \textit{Virtual Air Guitar} \cite{maki2005} are all recreations of existing instruments in a virtual world. 
Another example is Mike Wozniewski's instrument. In his program, the user can navigate in a 3D scene where some precise points generate sounds\cite{wozniewski2006spatial}. The user hears the sounds according to his position and orientation in the scene.

The aerial percussion is a 3D instrument that could be in this instrument group.

\subsection{The Drile}
The Drile offers a new usage of 3D. It uses 3D interaction to manipulate more easily the internal structure of a music.

Both the Drile and the aerial percussion were built for musical performance.

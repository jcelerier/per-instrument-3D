\section{History of the 3D musical instruments}
\paragraph{}
// TRAD A VERIFIER
De nombreux instruments de musiques immersifs se concentrent sur la navigation dans un environnements 3D virtuelle.
Tout d'abord le projet Phase \cite{rodet2005study} explore la génération, la prise en main et le controle de son ou de musique à l'aide d'un capteur haptique et d'une représentation visuelle pouvant guidée l'utilisateur.
Un second projet, Plumage \cite{plumage2007}, est une interface pour le contrôle interactif de la composition audio spatialisées. Des plumes dispersées dans une scène 3D représente des grains sonores, générent du son lorsque des têtes de lectures les parcours. Les têtes de lectures sont contrôlées directement par l'utilisateur.
Néanmoins ces deux projets ne permettent pas de manipuler directement la structure de la synthèse sonore, mais seulement de la manipuler.

Many immersive musical instrument tends to focus on navigation in a 3D virtual environment.
First of all, there is the project Phase \cite{rodet2005study} which explores the generation, the handling and the control of sound (or music) through an haptic sensor and a visual representation that guides the user.
Secondly, there is Plumage \cite{plumage2007} which is an interface for interactive control of spatialized audio composition. Feathers are scattered in a 3D scene which represents sound grains. They generate sound when they are crossed by reading heads. Those heads are controlled by the user.
Nevertheless those two projects do not allow to manipulate directly the sound synthesis structure but only existing sounds. 

\paragraph{}
Une autre gamme d'instrument 3D se concentre sur une unique synthèse sonore. Dans ce cas nous pouvons trouver par exemple le \textit{Virtual Xylophone}, le \textit{Virutal Membrane} ou encore la \textit{Virtual Air Guitar} \cite{maki2005}. Un autre exemple d'interaction 3D avec une synthèse sonore unique est celle de Mike Wozniewski \cite{wozniewski2006spatial}. Son application permet a un utilisateur de naviguer dans une scène 3D comportant à certain point précis des générations de son. L'utilisateur entend les sons en fonction de sa position et de son orientation dans la scène 3D.

Another range of 3D instruments centres on one single kind of sound synthesis. For example we can found the \textit{Virtual Xylophone}, the \textit{Virutal Membrane} or the \textit{Virtual Air Guitar} \cite{maki2005}. Another example is Mike Wozniewski's \cite{wozniewski2006spatial} instrument. In his program, the user can navigate in a 3D scene where some precise points generate sounds. The user hears the sounds according to his position and orientation in the scene.
//
\\
La percussion aérienne est un intstrument 3D que nous pourrions mettre dans cette classe d'instrument.

The aerial percussion is a 3D instrument that we could put in this instrument group.

\paragraph{}
Le DRILE propose une nouvelle utilisation de la 3D. Le DRILE utilise l'interaction 3D pour pouvoir manipuler plus aisément la structure même d'une musique.
\\
The DRILE offers a new usage of 3D. It uses 3D interaction to manipulate more easily the internal structure of a music.

Le DRILE et la percussion aérienne ont été conçu pour la performance musicale.

The DRILE and the aerial percussion were built for musical performance.
//

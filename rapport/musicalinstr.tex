\section{History of the 3D musical instruments}
\paragraph{}
De nombreux instruments de musiques immersifs se concentrent sur la navigation dans un environnements 3D virtuelle.
Tout d'abord le projet Phase \cite{rodet2005study} explore la génération, la prise en main et le controle de son ou de musique à l'aide d'un capteur haptique et d'une représentation visuelle pouvant guidée l'utilisateur.
Un second projet, Plumage \cite{plumage2007}, est une interface pour le contrôle interactif de la composition audio spatialisées. Des plumes dispersées dans une scène 3D représente des grains sonores, générent du son lorsque des têtes de lectures les parcours. Les têtes de lectures sont contrôlées directement par l'utilisateur.
Néanmoins ces deux projets ne permettent pas de manipuler directement la structure de la synthèse sonore, mais seulement de la manipuler.
\paragraph{}
Une autre gamme d'instrument 3D se concentre sur une unique synthèse sonore. Dans ce cas nous pouvons trouver par exemple le \textit{Virtual Xylophone}, le \textit{Virutal Membrane} ou encore la \textit{Virtual Air Guitar} \cite{maki2005}. Un autre exemple d'interaction 3D avec une synthèse sonore unique est celle de Mike Wozniewski \cite{wozniewski2006spatial}. Son application permet a un utilisateur de naviguer dans une scène 3D comportant à certain point précis des générations de son. L'utilisateur entend les sons en fonction de sa position et de son orientation dans la scène 3D.
\\
La percussion aérienne est un intstrument 3D que nous pourrions mettre dans cette classe d'instrument.
\paragraph{}
Le DRILE propose une nouvelle utilisation de la 3D. Le DRILE utilise l'interaction 3D pour pouvoir manipuler plus aisément la structure même d'une musique.
\\
Le DRILE et la percussion aérienne ont été conçu pour la performance musicale.

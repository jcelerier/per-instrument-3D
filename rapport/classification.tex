\section{Definition of a 3D display}
While it is commonplace to hear about 3D display in television or smartphone advertisement nowadays, the distinction between 2D and 3D might be more difficult to settle.
\subsection{The problem}
If we take the simple definition : a 3D display is a display that can show 3D images, it is really ambiguous, because of what is supposed to be "3D". For instance, for years, video games have been advertising 3D engines and spectacular 3D graphics, even without what we now call 3D displays.

Hence, we have to qualify what is 3D and what is not.
\subsection{Parameters}
In the litterature (\cite{okoshi1976three}, \cite{pimenta2012comprehensive}), the main idea is to relate to the human brain and body capabilities. For instance, a big part of the "3D" feel is due to the fact of having two eyes that looks in the same direction, but from a slightly different angle, but it is not all.

The visual cues of 3D vision are separated in two families:  
\begin{itemize}
\item Physiological cues. They will relate to the capabilities of the human body.
\item Psychological cues. They will relate to the information inference capabilities of the human brain.
\end{itemize}
\subsection{Presentation of common visual cues}

\begin{figure}[h!]
\centering
\begin{tabular}{|c|}
\hline
Psychological cues \\
\hline
\Gls{occlusion} \\
\Gls{linpersp} \\
\Gls{atmpersp} \\
Shading \\
\hline
\end{tabular}
\caption{Psychological cues}
\end{figure}
\section{Classification of the 3D displays}
One of the main problem while trying to find a proper \gls{display} for a given application is to choose a relevant classification for the displays, that allows a choice with criterions relevant to the application.
\subsection{Criterions}
There was a lack of proper nomenclature in the literature for a long time \cite{pimenta2012comprehensive}. However, some attemps have been made to find relevant criterions that would be general enough to cover the current display techniques, but also the ones that are not yet thought of.

\subsubsection{Different classifications}
The first classification was in \cite{okoshi1976three}, and it was really based upon the different kinds of displays : 
\begin{itemize}
\item Lens-sheet three dimensional pictures.
\item Projection-type three dimensional displays.
\item Holography.
\end{itemize}

However, it did not hold well against the emergence of new techniques, like volumetric displays for instance.

Other classifications \cite{ref nécessaire} would limit themselves to only a subset of 3D displays.

Hence the need for a classification that would not base itself on the different technologies, but on criterions that would be inherent to the idea of display and human vision.

\subsubsection{Chosen classification}
In \cite{pimenta2012comprehensive}, the main idea is to classify the displays according to two axes : 

\begin{itemize}
\item The display depth (flat or deep).
\item The number of points of view from which the image can be seen (duoscopic, multiscopic, or omniscopic).
\end{itemize}

\section{In-depth presentation of some 3D display methods}
\subsubsection{Pepper's Ghost}
\subsubsection{Glasses}
\subsubsection{Head-mounted displays}
\subsubsection{Hologram}
\subsubsection{Autostereoscopic screen}
% En rajouter à volonté
\chapter{3D musical instruments and 3D displays for performance}
\section{Présentation du sujet}
\paragraph{}
Au Scrime et au Labri, des instruments de musique 3D ont été développé dans le cadre de recherche dans le monde de la réalité virtuelle interactif et de la musique informatique. 
\\
Le DRILE \cite{berthaut2010drile} est une instrument de musique 3D qui permet de manipuler la structure d'une musique basée sur le principe du \textit{live looping} dans une scène de réalité virtuelle immersive.
\\
La percussion aérienne est un instrument de musique 3D qui permet de générer du son à l'aide de capteurs situés aux bouts de baguettes de batteries. Des formes géométriques 3D virtuelles sont situées autours du musiciens. En fonction de la position des capteurs, de l'orientation, et de la vitesse, des sons sont générés.

\paragraph{}
Dans ce cadre là, il nous a été demandé de mettre en place un prototype de dispositif de rendu 3D pour la performance musicale. Il est nécessaire de tenir compte des contraintes liées à la performance, ainsi que les contraintes liées aux instruments.
\\
Les contraintes de la performance sont:
\begin{itemize}
\item Le musicien doit être face aux spectateurs
\item Le musicien doit avoir les informations nécessaires à l'utilisation de son instrument, ainsi que les spectateurs doivent avoir une représentation de l'instrument pour comprendre les actions du musicien
\end{itemize}

\paragraph{}
Pour ce faire, la description d'un instrument de musique 3D nécessite d'être préciser.

\newpage
\section{What is an 3D musical instrument?}

\paragraph{}

A 3D musical instrument, or an immersive virtual musical instrument, represents sound processes and their parameters as 3D entities of a virtual reality so that they can be perceived not only through auditory feedback but also visually in 3D and possibly through tactile as well as haptic feedback, using 3D interface metaphors consisting of interaction techniques such as navigation, selection and manipulation.

\paragraph{}

Par exemple, nous pouvons voir sur la photographie \ref{drile} que l'utilisateur est équipé de lunette et de manette avec capteurs à retour haptiques (Piivert \cite{berthaut2010piivert}). L'utilisateur manipule des forme 3D dans un environnement 3D pour influer sur la génération de la musique. L'instrument en question est le DRILE. Nous vous donnerons plus d'informations sur cet instrument dans la suite du rapport.

\begin{figure}[t]
\centering
\includegraphics[scale=0.3]{image/drile.jpg}
\caption{Picture of a musician using DRILE}
\label{drile}
\end{figure}

\paragraph{}
Maintenant qu'un instrument de musique 3D a été défini, le concept d'immersion et d'interactivité doivent être définis.
\newpage
\section{Immersion}

\paragraph{}
L'immersion est un état psychologique où le sujet cesse de se rendre comtpe de son propre état physique. Le concept d'immersion est très important en réalité virtuelle. Par exemple pour la percussion aérienne cela reviendrait à l'état atteind par le musicien lorsque celui ci ne réfléchit plus consciemment à la disposition des formes avec lesquelles il interragit. Le musicien sera alors immergé dans cette scène 3D que constitue la disposition des formes de la percussion aérienne.

\paragraph{}
Ainsi pour immerger un utilisateur, il est possible de jouer sur plusieurs paramètres. Ceux ci sont très liés aux sens du corps humain. Dans le cadre de notre projet nous nous intéresserons surtout à la vision, et tout particulièrement aux dispositifs de rendu de la 3D.



\section{Control}
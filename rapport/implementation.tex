\section{Implementation}
We will describe here the multiple choices that have been made during this project, and the reason behind these choices, as well as the result of our implementation.

\subsection{Chosen display techniques}
There are multiple factors to take into account : 
\begin{enumerate}
\item The availability of the technology.
\item The potential price of the required materials.
\item The time to setup the display.
\item The scaling for a medium-sized audience.
\item The compatibility with the double requirement : a view for the performer, and another for the spectators.
\end{enumerate}

We are now going to study these requirements point by point.
\subsubsection{Availability}
This is the main problem : many of the display devices presented in \fref{chap:3ddisp} have only been the subject of research and not of a real implementation sold by a company (e.g. holograms). Also, the development state of some technologies  might not be sufficient for what we are striving for (e.g. autostereoscopic displays which are only present in very small screens like smartphones).
\subsubsection{Price}
Some technologies might be irrelevant only because of the amount of money needed to get a working implementation. For instance, an active 84" 3D HDTV generally costs more than ten thousand dollars, which is unsuitable to this project.
\subsubsection{Setup time}
Some methods might require a very long time to setup. While we don't have a required maximum time to setup the show, we should try to keep it as low as possible. For instance, the Pepper's Ghost technique is quite long to setup, because there is a lot of massive hardware, videoprojectors, screens, to setup.
\subsubsection{Scaling}
Since this is for a show, we need a system that will allow everybody in the room to enjoy the performance. The estimate is at about 40 persons : we need a display that provides big enough viewing angles and is big enough for everybody to be able to enjoy it. A square display with a side of two meters would be ideal to enable complete immersion.
\subsubsection{Double-view requirement}
This is one of the hardest requirements, because it can easily double the quantity of required hardware. For instance, if we were to use 3D TVs, we would need one TV for the viewers and one for the performer.

\subsection{DRILE Implementation}
\subsubsection{Technologies used}
\subsubsection{Pictures}
\subsection{Aerial Percussion implementation}
\subsubsection{Technologies used}
\subsubsection{Displaying the data}
\paragraph{Truc sur les angles}
\subsubsection{Pictures}
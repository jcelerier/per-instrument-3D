\newglossaryentry{display}
{
  name=display,
  description={a visual output device for the presentation of images \cite{pimenta2012comprehensive}}
}
\newglossaryentry{stereopsis}
{
name=stereopsis,
description={The ability of the brain to create a single mental image from two eyes}
}
\newglossaryentry{occlusion}
{
	name=occlusion,
	description={when an object is hidden by another}
}

\newglossaryentry{linpersp}
{
	name={linear perspective},
	description={the way our visual perception of objects are affected by their position and dimension}
}

\newglossaryentry{atmpersp}
{
	name={atmospheric perspective},
	description={the impression of depth given by the refraction of the air. For instance, we can say that mountains are far because they appear more blue than close mountains}
}	

\newglossaryentry{shading}
{
	name=shading,
	description={the gradient in color and shades that would appear due to the shape of objects and color, intensity, and direction of light}
}

\newglossaryentry{motionparallax}
{
	name={motion parallax},
	description={when two objects, one further from another, seem to translate at a different speed if the observer is moving}
}
\newglossaryentry{kineticdepth}
{
	name={kinetic depth},
	description={the visual cues we have of an object in motion}
}

\newglossaryentry{stereoscopy}
{
	name=stereoscopy,
	description={the result of the human eyes receiving two different images}
}

\newglossaryentry{accomodation}
{
	name=accomodation,
	description={the change of focus of the eyes in order to perceive clearly what is looked at}
}

\newglossaryentry{convergence}
{
	name=convergence,
	description={when the eyes rotate to aim on the point in space a person focuses on}
}


\newglossaryentry{livelooping}
{
	name={live looping},
	description={A REMPLIR}
}
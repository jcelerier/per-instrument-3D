\section{Required work}
Apart from the research work, the application of our research to two musical instruments (the DRILE and the Aerial Percussion) is required.

The goal of our work is to enact a live show with these two instruments, that allows for both the performer and the spectators to see the musical instrument in three dimensions.

\subsection{Finding a display}
The first task is to find a suitable display method that would allow : 
\begin{itemize}
\item The performer to interact with the instrument
\item The spectators to see the performer as if he was part of the 3D scene
\item If possible, a stereoscopic feel.
\end{itemize}

\subsection{Implementing suitable renderings}
There is already some existing work for the rendering engine of the DRILE, however there is nothing for the aerial percussion.

We have to make renders from two different viewpoints : one for the performer, antoher for the spectators.

\subsection{Customisation}
If we have time left, we are to add some customisations to the aerial percussion rendering, in order to make it look like a real show, with special effects, flares, particles, textures...
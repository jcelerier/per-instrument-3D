\section{Objectives}
Apart from the research work, it is asked from us to apply our research to two digital musical instruments (the \brand{Drile} and the Aerial Percussion).

The goal of our work is to enact a live show with these two instruments, that allows for both the performer and the spectators to see the musical instrument in three dimensions.

\subsection{Finding a display}
The first task is to find a suitable display method that would allow : 
\begin{itemize}
\item The performer to interact with the instrument
\item The spectators to see the performer as if he was part of the 3D scene
\item If possible, a stereoscopic feel.
\end{itemize}

\subsection{Implementing suitable renderings}
There is already some existing work for the rendering engine of the Drile, however there is nothing for the aerial percussion.

We have to make renders from two different viewpoints : one for the performer, another for the spectators.

\subsection{Customization}
If we have time left, we are to add some customizations to the aerial percussion rendering, in order to make it look like a real show, with special effects, flares, particles, textures...
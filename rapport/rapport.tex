\documentclass[a4paper,11pt,oneside]{report}

\usepackage{geometry}
\usepackage[utf8]{inputenc}
\usepackage[english]{babel}
\usepackage[T1]{fontenc}
\usepackage{relsize}
\usepackage{color}

\usepackage{hyperref}

\definecolor{dkgreen}{rgb}{0,0.6,0}
\definecolor{gray}{rgb}{0.5,0.5,0.5}
\definecolor{mauve}{rgb}{0.58,0,0.82}

\usepackage{listings}
\usepackage{float}
\usepackage{kpfonts}
\usepackage{verbatimbox}
\usepackage{datetime}

% more figures per page
\renewcommand\floatpagefraction{.9}
\renewcommand\topfraction{.9}
\renewcommand\bottomfraction{.9}
\renewcommand\textfraction{.1}   
\setcounter{totalnumber}{50}
\setcounter{topnumber}{50}
\setcounter{bottomnumber}{50}

\usepackage{graphicx}
%% \newcommand{\foofoo}{\hspace{-2.3pt}$\bullet$ \hspace{5pt}}


\lstset{
language=C++,
basicstyle=\footnotesize,
numbers=left,
numberstyle=\footnotesize,
stepnumber=1,
numbersep=5pt,
backgroundcolor=\color{white},
showspaces=false,
showstringspaces=false,
showtabs=false,
frame=single,
tabsize=2,
captionpos=b,
breaklines=true,
breakatwhitespace=false,
escapeinside={\%*}{*)}
}


\lstset{
  literate={ù}{{\`u}}1
           {é}{{\'e}}1
           {è}{{\'e}}1
           {à}{{\`a}}1
}


\geometry{margin=2cm}
\geometry{headheight=15pt}
\usepackage{fancyhdr}
\usepackage{fancyref}
\usepackage[xindy,toc]{glossaries}
\usepackage{natbib}
\usepackage{fancyvrb}
\usepackage{float}
\usepackage{algorithm2e}
\usepackage{CJKutf8}
\usepackage{caption}
\usepackage[footnote,smaller]{acronym}
\usepackage[perpage]{footmisc}
\usepackage[]{titlesec}
\makeatletter
\def\ttl@mkchap@i#1#2#3#4#5#6#7{%
    \ttl@assign\@tempskipa#3\relax\beforetitleunit
    \vspace{\@tempskipa}
    \global\@afterindenttrue
    \ifcase#5 \global\@afterindentfalse\fi
    \ttl@assign\@tempskipb#4\relax\aftertitleunit
    \ttl@topmode{\@tempskipb}{%
        \ttl@select{#6}{#1}{#2}{#7}}%
    \ttl@finmarks  % Outside the box!
    \@ifundefined{ttlp@#6}{}{\ttlp@write{#6}}}
\makeatother

\makeglossaries

\pagestyle{fancy}

\renewcommand*{\acsfont}[1]{\brand{#1}}
\newcommand{\brand}[1]{\textsc{\textbf{#1}}}

\let\oldpar\paragraph
\renewcommand{\paragraph}[1]{\oldpar{#1}\mbox{}\\}

\DeclareMathOperator{\power}{power}
\DeclareMathOperator{\phase}{phase}
\DeclareMathOperator{\recompute}{recompute}

\rhead{3D Musical instrument}

\acrodef{LABRI}{LAboratoire Bordelais de Recherche en Informatique}
\acrodef{SCRIME}{Studio de Création et de Recherche en Informatique et Musique Électroacoustique}

\begin{document}
\selectlanguage{english}
\begin{titlepage}
  \begin{center}

    \textsc{\LARGE Three-dimensional musical instrument}\\[1cm]
    
  \end{center}
  
  \vspace*{\stretch{2}}
  \begin{flushbottom}
   \begin{flushleft}
    \underline{Students} : Mohamed \textsc{Bourara}, Jean \textsc{Bui-Quang}, Jean-Michaël \textsc{Celerier}, Damien \textsc{Clergeaud}, Marie \textsc{Immacula Omiscar}, Omar \textsc{Ourhi} - \today \\
   \end{flushleft}
  \end{flushbottom}
\end{titlepage}
\clearpage
\tableofcontents
\listoffigures
\newglossaryentry{display}
{
  name=display,
  description={a visual output device for the presentation of images \cite{pimenta2012comprehensive}}
}
\newglossaryentry{interactionplane}
{
	name={interaction plane},
	description={the plane on which the shapes of the Drile are mutable}
}
\newglossaryentry{stereopsis}
{
name=stereopsis,
description={The ability of the brain to create a single mental image from two eyes}
}
\newglossaryentry{occlusion}
{
	name=occlusion,
	description={when an object is hidden by another}
}

\newglossaryentry{linpersp}
{
	name={linear perspective},
	description={the way our visual perception of objects are affected by their position and dimension}
}

\newglossaryentry{atmpersp}
{
	name={atmospheric perspective},
	description={the impression of depth given by the refraction of the air. For instance, we can say that mountains are far because they appear more blue than close mountains}
}	

\newglossaryentry{shading}
{
	name=shading,
	description={the gradient in color and shades that would appear due to the shape of objects and color, intensity, and direction of light}
}

\newglossaryentry{motionparallax}
{
	name={motion parallax},
	description={when two objects, one further from another, seem to translate at a different speed if the observer is moving}
}
\newglossaryentry{kineticdepth}
{
	name={kinetic depth},
	description={the visual cues we have of an object in motion}
}

\newglossaryentry{stereoscopy}
{
	name=stereoscopy,
	description={the result of the human eyes receiving two different images}
}

\newglossaryentry{accomodation}
{
	name=accomodation,
	description={the change of focus of the eyes in order to perceive clearly what is looked at}
}

\newglossaryentry{convergence}
{
	name=convergence,
	description={when the eyes rotate to aim on the point in space a person focuses on}
}


\newglossaryentry{livelooping}
{
	name={live looping},
	description={A REMPLIR}
}

\chapter{Introduction}
A three-dimensional musical instrument might sound quite abstract for the bystander. One can think of it as a musical instrument taking place in the virtual reality or augmented reality domain. While an exact definition might be hard to settle because every instrument will be different in core features to others, a general definition might be an instrument which can have either :
\begin{itemize}
\item A visual representation in a three-dimensional space
\item Interactions in a three-dimensional space
\end{itemize}

The two points are generally shared, however it is harder to display the instrument in 3D than to interact with it.

The display can have two goals :  \cite{berthaut2010piivert}
\begin{itemize}
\item Giving visual cues to the spectators of the musician's actions.
\item Helping the musician to perform.
\end{itemize}

One of the main focuses of this exposé will be to assess the different display techniques suitables to a 3D instrument, and the other will explain how it can improve existing 3D instruments.

The last part of this report will be about the choices we had to make in order to setup our own 3D musical instrument.

\chapter{3D musical instruments and 3D displays for performance}
\section{Présentation du sujet}
\paragraph{}
Au Scrime et au Labri, des instruments de musique 3D ont été développé dans le cadre de recherche dans le monde de la réalité virtuelle interactif et de la musique informatique. 
\\
Le DRILE \cite{berthaut2010drile} est une instrument de musique 3D qui permet de manipuler la structure d'une musique basée sur le principe du \textit{live looping} dans une scène de réalité virtuelle immersive.
\\
La percussion aérienne est un instrument de musique 3D qui permet de générer du son à l'aide de capteurs situés aux bouts de baguettes de batteries. Des formes géométriques 3D virtuelles sont situées autours du musiciens. En fonction de la position des capteurs, de l'orientation, et de la vitesse, des sons sont générés.

\paragraph{}
Dans ce cadre là, il nous a été demandé de mettre en place un prototype de dispositif de rendu 3D pour la performance musicale. Il est nécessaire de tenir compte des contraintes liées à la performance, ainsi que les contraintes liées aux instruments.
\\
Les contraintes de la performance sont:
\begin{itemize}
\item Le musicien doit être face aux spectateurs
\item Le musicien doit avoir les informations nécessaires à l'utilisation de son instrument, ainsi que les spectateurs doivent avoir une représentation de l'instrument pour comprendre les actions du musicien
\end{itemize}

\paragraph{}
Pour ce faire, la description d'un instrument de musique 3D nécessite d'être préciser.

\newpage
\section{What is an 3D musical instrument?}

\paragraph{}

A 3D musical instrument, or an immersive virtual musical instrument, represents sound processes and their parameters as 3D entities of a virtual reality so that they can be perceived not only through auditory feedback but also visually in 3D and possibly through tactile as well as haptic feedback, using 3D interface metaphors consisting of interaction techniques such as navigation, selection and manipulation.

\paragraph{}

Par exemple, nous pouvons voir sur la photographie \ref{drile} que l'utilisateur est équipé de lunette et de manette avec capteurs à retour haptiques (Piivert \cite{berthaut2010piivert}). L'utilisateur manipule des forme 3D dans un environnement 3D pour influer sur la génération de la musique. L'instrument en question est le DRILE. Nous vous donnerons plus d'informations sur cet instrument dans la suite du rapport.

\begin{figure}[t]
\centering
\includegraphics[scale=0.3]{image/drile.jpg}
\caption{Picture of a musician using DRILE}
\label{drile}
\end{figure}

\paragraph{}
Maintenant qu'un instrument de musique 3D a été défini, le concept d'immersion et d'interactivité doivent être définis.
\newpage
\section{Immersion}

\paragraph{}
L'immersion est un état psychologique où le sujet cesse de se rendre comtpe de son propre état physique. Le concept d'immersion est très important en réalité virtuelle. Par exemple pour la percussion aérienne cela reviendrait à l'état atteind par le musicien lorsque celui ci ne réfléchit plus consciemment à la disposition des formes avec lesquelles il interragit. Le musicien sera alors immergé dans cette scène 3D que constitue la disposition des formes de la percussion aérienne.

\paragraph{}
Ainsi pour immerger un utilisateur, il est possible de jouer sur plusieurs paramètres. Ceux ci sont très liés aux sens du corps humain. Dans le cadre de notre projet nous nous intéresserons surtout à la vision, et tout particulièrement aux dispositifs de rendu de la 3D.



\section{Control} % Relation avec instrus 3D
\chapter{Three-dimensional displays}
\label{chap:3ddisp}
\section{Classification of the 3D displays}
\subsection{Criterions}
\subsection{Presentation of the visual cues}
\subsection{In-depth presentation of some 3D display methods}
\subsubsection{Pepper's Ghost}
\subsubsection{Glasses}
\subsubsection{Head-mounted displays}
\subsubsection{Hologram}
\subsubsection{Autostereoscopic screen}
% En rajouter à volonté % Explication des classifications
\chapter{Presentation of 3D musical instruments}
\section{History of the 3D musical instruments}
\subsection{3D Navigation instruments}
Many immersive musical instruments tend to focus on navigation in a 3D virtual environment.
First of all, there is the Phase project  \cite{rodet2005study} which explores generation, handling and control of sound (or music), through a haptic sensor and a visual representation that guides the user.
Secondly, there is Plumage \cite{plumage2007} which is an interface for interactive control of \glslink{spatialization}{spatialized} audio composition. Feathers are scattered in a 3D scene which represents sound grains. They generate sound when they are crossed by reading heads. Those heads are controlled by the user.
Nevertheless these two projects do not allow manipulation of direct sound synthesis structure, only of existing sounds. 

\subsection{Single sound synthesis instruments}
Another range of 3D instruments focuses on a single kind of sound synthesis. For instance, the \textit{Virtual Xylophone}, the \textit{Virtual Membrane} or the \textit{Virtual Air Guitar} \cite{maki2005} are all recreations of existing instruments in a virtual world. 
Another example is Mike Wozniewski's instrument. In his program, the user can navigate in a 3D scene where some precise points generate sounds\cite{wozniewski2006spatial}. The user hears the sounds according to his position and orientation in the scene.

The aerial percussion is a 3D instrument that could be in this instrument group.

\subsection{The Drile}
The Drile offers a new usage of 3D. It uses 3D interaction to manipulate more easily the internal structure of a music.

Both the Drile and the aerial percussion were built for musical performance.

\section{Le DRILE}
\begin{frame}
  DRILE DRILE DRILE
\end{frame}

\section{Aerial Percussion}

\paragraph{}
La percussion aérienne est un instrument de musique innovant par rapport au percussion aérienne car elle ne comporte pas les fûts qu'il faut percuter pour générer le son. La position des baguettes dans l'espace est mesurée à chaque instant grâce à des capteurs. Les données obtenues sont traitées par un logiciel qui permet de détecter les impacts en temps réel et de leur associer des sons. L'utilisation d'un ordinateur permet d'obtenir une palette sonore sans limite. Elle permet aussi au musicien de s'abstraire des contraintes physique de l'instrument, il peut chorégraphier ses mouvements, la mise en scène gagne de l'importance.
Mais comment fonctionne techniquement cette instrument?

\subsection{Polhemus Liberty}

\paragraph{}
Le Polhemus Liberty est un dispositif qui permet de connaître la position et l'orientation de capteurs dans l'espace en temps réel. Un émetteur permet de générer un champ magnétique à l'aide de trois antennes fixes placées orthogonalement les unes par rapport aux autres dans un cube d'une dizaine de centimètres de côté. Ce cube, constituant la base émettrice, est relié au bloc central par un câble et est placé idéalement au centre de la zone dans laquelle les acquisitions sont faites. Le champ magnétique diffusé est capté par un second ensemble d'antennes que constitues chacun des capteurs. Ceux-ci se présentent sous la forme de petits cubes de moins d'un centimètre de côté placés au bout des câbles qui les relient au bloc central. Le signal reçu par les capteurs est ensuite échantillonné, traité pas des processeurs DSP (Digital Signal Processor) dans le bloc central, puis envoyé à l'ordinateur via celui-ci par port USB 2.0.

Tous les composants de la percussion sont sur la figure \ref{fig:percu}, de gauche à droite il y a les baguettes, le bloc central et la base émettrice.

\begin{figure}[h!]
\centering\includegraphics[scale=0.11]{image/percu.jpg}
\caption{Polhemus Liberty and sticks.}
\label{fig:percu}
\end{figure}

\paragraph{}
Mais le dispositif ne fonctionne pas seul. Il y a aussi une couche logiciel.

\subsection{SetKreator and FoB}

SetKreator a été développé au \ac{SCRIME} par Joseph Larralde et Sébastien Lebreton. SetKreator est un éditeur d'instruments virtuels pour la percussion aérienne. Il permet de créer des volumes basiques (parallépipèdes, cylindres...) à l'intérieur d'une sphère représentant la zone d'utilisation des capteurs Polhemus, comme à la figure \ref{fig:setkreator}. Il permet aussi d'associer à chaque forme une synthèse sonore particuliére. Il reçoit les données de position des baguettes par flux OSC (Open Sound Control).

\begin{figure}[h!]
\centering\includegraphics[scale=0.3]{image/setkreator.png}
\caption{Interface of SetKreator}
\label{fig:setkreator}
\end{figure}


Pour faire le lien entre SetKreator et le Polhemus, une autre programme est utile. Il s'agit de FoB. FoB est une petit programme qui reçoit les données de positions du Polhemus via le réseau local de l'ordinateur. Et les renvoie par flux OSC.



\chapter{Realisation}
\section{Required work}
\subsection{Finding a display}
\subsection{Implementing suitable renderings}
\subsection{Bells \& whistles} % c'est pas sérieux hein
\section{Implémentation}
\begin{frame}
  Implémentation ici
\end{frame}


\section{Conclusion}
\begin{frame}
  \begin{itemize}
  \item We are currently working on the implementation.
  \item Some tests already realised with headtracking and 3D rendering library (openFrameworks).
  \item The chosen technology is the Pepper Ghost technique, with a wavelength selective display if it is available.
  \end{itemize}
\end{frame}


\printglossaries
\bibliographystyle{alpha}
\bibliography{instru3d}
\end{document}

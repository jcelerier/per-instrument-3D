\subsection{livelooping}
L'instrument Drile qui constitue une partie importante de notre étude, est basé sur la technique live-looping hiérarchique, il est donc bon de parler de cette technique.\\
Tout d'abord le live-looping classique est une technique qui consiste à enregistrer des échantillons audio ou de contrôle, et de lire cet enregistrement en boucle, en temps réel . L'idée de base de cette technique est de prendre (capturer) et de mettre en boucle, une partie d'une performance ou d'une chanson en temps réel puis de pouvoir jouer d'autres morceaux par dessus. \\
Le live-looping hiérarchique est plus adapté pour obtenir de bonnes structures musicales, il permet d'améliorer le live-looping classique en ajoutant une structure arborescente à ce dernier. Un arbre est composé de nœuds, de feuilles et d’enfants. Dans cette structure arborescente que propose le live-looping hiérarchique, les feuilles contiennent tout ce qui est audio et les nœuds des événements de contrôle. Pour les deux techniques, les effets qu'on peut ajouter à une composition musicale sont pratiquement identiques sauf que, pour les nœuds il y a en plus les effets de contrôle.\\
Il existe un ensemble d'opérations qui permet de manipuler ces arbres. La possibilité de fusionner ces différentes opérations, facilite la construction de structures musicales variées et complexes, contrairement à la technique live-looping classique.\\

The Drile instrument which is an important part of our study is based on the technique hierarchical live- looping , so it is good to talk about this technique. First of all the classic live- looping is a technique that consists in recording audio samples or control, and read this loop recording in real time . The basic idea of this technique is to take ( capture ) and to loop part of a performance or song in real time and be able to play other pieces on it. The hierarchical live- looping is more suitable for good musical structures , it improves the classic live- looping by adding a tree structure on it. A tree consists of nodes , leaves, and children. In this tree structure that provides the hierarchical live- looping , the leaves contain everything audio and nodes contain control events . For both techniques , the effects that can be added to a musical composition are substantially identical except that , for the nodes there are additionnal control effects . There is a set of operations to manipulate these trees. The possibility of merging these operations , facilitates the construction of musical structures varied and complex , unlike the conventional technique live- looping .
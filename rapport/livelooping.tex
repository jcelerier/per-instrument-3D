\subsection{Hierarchical Live-Looping}
The Drile musical process is based on the hierarchical live-looping technique.

\subsubsection{Live-Looping}
Live-looping is a composition technique which came to light thanks to famous modern composers such as Steve Reich.
The concept is relatively simple : music  is composed using only samples that are played in loops. Stacking those loops and adding multiple audio effects on them can easily produce complex musics.
It has been used a lot for musical performance especially by singers, guitarists because they can build their own backing tracks easily. Many electronic musicians are using this technique to produce entire songs.

Nevertheless, one of its limits is that one cannot manipulate complex structures but only sequences of samples. So the structure remains quite linear.

\subsubsection{Hierarchical Live-Looping}
Inspired from Marczak \cite{marczak2007etude} works, hierarchical live-looping organizes live-looping in a tree structure which leads to a more complex musical structure. 

Leafs are composed of raw samples (i.e audio extract or synthesized sound) and a musical effects list that is to be applied on the associated sample. 
Then, each node is made of a node children list, a musical effects list, and a musical content. 
This last content is the result of the children musical contents.

In that way, adding an effect to a leaf will only affects its own musical content. Whereas adding an effect to a node will apply the effects to the whole musical content produced from all children. It becomes possible with operations on tree (duplication, merging\dots ) to manipulate a more sophisticated musical structure. 

All these tree manipulations get simplified with the 3D representation of this structure.
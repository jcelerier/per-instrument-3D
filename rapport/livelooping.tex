\subsection{Livelooping}
L'instrument Drile qui constitue une partie importante de notre étude, est basé sur la technique live-looping hiérarchique, il est donc bon de parler de cette technique.\\
Tout d'abord le live-looping classique est une technique qui consiste à enregistrer des échantillons audio ou de contrôle, et de lire cet enregistrement en boucle, en temps réel . L'idée de base de cette technique est de prendre (capturer) et de mettre en boucle, une partie d'une performance ou d'une chanson en temps réel puis de pouvoir jouer d'autres morceaux par dessus. \\
Le live-looping hiérarchique est plus adapté pour obtenir de bonnes structures musicales, il permet d'améliorer le live-looping classique en ajoutant une structure arborescente à ce dernier. Un arbre est composé de nœuds, de feuilles et d’enfants. Dans cette structure arborescente que propose le live-looping hiérarchique, les feuilles contiennent tout ce qui est audio et les nœuds des événements de contrôle. Pour les deux techniques, les effets qu'on peut ajouter à une composition musicale sont pratiquement identiques sauf que, pour les nœuds il y a en plus les effets de contrôle.\\
Il existe un ensemble d'opérations qui permet de manipuler ces arbres. La possibilité de fusionner ces différentes opérations, facilite la construction de structures musicales variées et complexes, contrairement à la technique live-looping classique.\\



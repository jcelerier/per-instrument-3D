\subsection{Second categorisation}
\begin{frame}
  \begin{itemize}
  \item Based on "A Comprehensive Taxonomy for Three-dimensional displays".

  \item Paper problematic : 

    \begin{center}
      Profusion of technologies $\implies$ Classification difficult.
    \end{center}

  \end{itemize}
  \begin{enumerate}
  \item First part : visual cues used by the human brain to define 3D vision.
  \item Second part : Definition of the properties of 3D screens.
  \item Third part : Presentation of the taxonomy created in this article.
  \end{enumerate}
\end{frame}

\subsection{Second categorisation: Visual cues}
\begin{frame}
  \begin{figure}
    \renewcommand{\arraystretch}{1.7}
    \begin{tabular}{cc}
      \begin{tabular}{|c|}
        \hline
            {\LARGE Physiological cues} \\
            \hline
            Binocular disparity \\
            Convergence \\
            Accommodation \\
            \hline
      \end{tabular} 
      &
      \begin{tabular}{|c|}
        \hline
            {\LARGE Psychological cues} \\
            \hline
                {\Large Static cues} \\
                \hline
                Shades and nuances \\
                Occlusion \\
                Perspective \\
                \hline
                    {\Large Dynamic cues} \\
                    \hline
                    Movement parallax \\
                    Cinetic depth \\
                    \hline
      \end{tabular}
    \end{tabular}
  \end{figure}
\end{frame}

\subsection{Second categorisation: 3D Display definition}
\begin{frame}
  \begin{center}
    \centering
    A 3D display makes use of at least one physiological cue.
  \end{center}

  Hence, it cannot be emulated strictly on the software side.
\end{frame}
\subsection{Second categorisation: 3D Display taxonomy}
\begin{frame}
  \begin{itemize}
  \item Two axes
    \begin{itemize}
    \item Number of views : Duoscopic, multiscopic, omniscopic
    \item Depth : Flat, deep
    \end{itemize}
  \item Two novel points in the article: 
    \begin{itemize}
    \item Multi-directional display: deep multiscopic.
    \item Virtual volume display: flat omniscopic.
    \end{itemize}
  \end{itemize}
\end{frame}
\subsection{Second categorisation: Novel points}
\begin{frame}
  \begin{enumerate}
  \item \Large{Virtual volume display}
    \begin{itemize}
    \item Either adaptative optics and Pepper Ghost derivatives or holographic systems : http://www.youtube.com/watch?v=Y1m7xEzlhWA.
    \item Only method that is able to present every single physiological cue.
    \end{itemize}
  \item \Large{Multi-directional display}
    \begin{itemize}
    \item Finite number of subdivisions but up to a $360\deg$ FoV.
    \item Two possibilities : 
      \begin{itemize}
      \item Rotative screen
      \item Multiple anisotropic screens. Light has to go in a single direction.
      \end{itemize}
    \end{itemize}
  \end{enumerate}
\end{frame}

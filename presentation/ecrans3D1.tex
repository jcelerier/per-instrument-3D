
	\section{Écrans 3D}
\subsection{Two-view 3D displays}
	\begin{frame}

	  \frametitle{Écrans 3D \\Two-view 3D displays} 
	 
				\begin{itemize}
				\item Wavelength Selective Displays:
				
				\begin{itemize}
				\item Chaque oeil reçoive l'image qui lui est destinée.
				\item Les images sont filtrées par la couleur.			
				\end{itemize}
					
			
			
			\item Avantage:
			\begin{itemize} 	
				\item Tout dispositif d'affichage de couleur peut être utilisé pour présenter la stéréoscopie.
					\end{itemize}
\item Inconvénient:
			\begin{itemize} 
				\item Chaque oeil voit une stimulus de couleur différente (le système visuel réagira face à une couleur)
		\end{itemize}
		\end{itemize}
	\includegraphics[keepaspectratio,height=.2\linewidth]{1.jpg}
	\end{frame}
	
	
	\begin{frame}		  	
	  \begin{itemize}
	  	\item Time-Sequential Two-View Displays:
        \begin{itemize}
				\item Time-Sequential Polarization:
		\begin{itemize}
		    \item Cette technique utilise des lunettes polarisées.
		    \item Le verre droit est polarisé dans un sens pendant que le verre gauche est polarisé dans l’autre    sens.
   		    \item L’image affichée sur l’écran est en fait constituée de deux images
		    \item Une ligne sur deux est donc destinée pour chaque œil.
		\end{itemize}
		\end{itemize}
\end{itemize}
\includegraphics[keepaspectratio,height=.2\linewidth]{2.jpg}
	\end{frame}

\begin{frame}
	  \begin{itemize}
	  	\item Time-Sequential Two-View Displays:
\begin{itemize}
				\item Time-Sequential Backlight:
		\begin{itemize}
		    \item Technique Autostereoscopique.
		    \item Rétroéclairage:technique d'éclairage par l'arrière.
   		    \item Avoir une source de lumière dans chaque côté de l'écran avec un guide d'onde surface entre eux.

		\end{itemize}
		\end{itemize}
\end{itemize}
\includegraphics[keepaspectratio,height=.2\linewidth]{3.jpg}
	\end{frame}
	
	\subsection{Horizontal parallax multiview 3D displays}	

	\begin{frame}

	  \frametitle{Écrans 3D \\Horizontal parallax multiview 3D displays} 
			
	\begin{itemize}
	\item Parallax Barrier Displays:
								
\begin{itemize}
\item C’est une technique Autostereoscopique.
\item Elle permet d'obtenir une vision relief sans le port de lunettes.
\end{itemize}
\item Les inconvénients:
\begin{itemize} 	
\item	Il faut se placer précisément par rapport à l’écran.
\item Il faut être stable.
\item Il ne permet pas la visualisation de l’image en relief à plusieurs spectateurs en même temps.
	\end{itemize}
		\end{itemize}
	\includegraphics[keepaspectratio,height=.2\linewidth]{4.png}
	\end{frame}
	
	\begin{frame}
\begin{itemize}
\item Multi-Projector Displays:\\
Cette technique consiste a positionner en cercle plusieurs vidéo-projecteurs affichant tous un angle d’image différent, apres ces images sont projetées sur un écran spécial. 						
\item Avantage:
\begin{itemize} 
\item Taille de l’image 3D peut être beaucoup plus grande
il n’est y a pas de limite.
	\end{itemize}
\item Les inconvénients:
\begin{itemize} 	
\item	Plusieurs projecteurs sont nécessaires (projecteur par vue)
\item Les projecteurs doivent être alignées avec précision.
	\end{itemize}
		\end{itemize}
	\end{frame}

\begin{frame}
	  \frametitle{Écrans 3D \\FULL PARALLAX MULTIVIEW 3D DISPLAYS} 
	 
Ce type d’affichage permet aux téléspectateurs de voir une scène en 3D de n'importe quel angle.\\
	
	\begin{itemize}
	\item Integral Imaging Displays:
								
\begin{itemize}
\item C’ est un mode d'affichage 3D auto-stéréoscopique,qui avait été initialement proposé par Lippmann en 1908.

\item C’une technique qui consiste a utiliser un réseau de micro-lentilles en face de l’image où chaque lentille est différente en fonction de l'angle de vision.

\end{itemize}
		\end{itemize}
		
	\end{frame}
	\begin{frame}
  \frametitle{Analyse} 
	
\begin{itemize}
\item Pour un affichage 3D:
\begin{itemize}
\item Position de l’œil
\item Résolution (pixels) par affichage de zone
\item Contraintes sur la position de la tête
\end{itemize}
\end{itemize}

\begin{itemize}
\item Domaine d’utilisation:
\begin{itemize}
\item Cinema
\item présentation de l'information et de la publicité
\item 3D pour les appareils portables
\end{itemize}
\end{itemize}
\begin{itemize}
\item Les technologies Stéréoscopique et Autostéréoscopique
\item Holographie
	\end{itemize}
\end{frame}


